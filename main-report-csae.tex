\documentclass[14pt, oneside]{altsu-report}

\worktype{Курсовая работа (2 курс)}
\title{Рост дендрита на Python}
\author{С.\,С.~Авдеев}
\groupnumber{5.205-2}
\GradebookNumber{1337}
\supervisor{И.\,А.~Шмаков}
\supervisordegree{Старший преподаватель}
\ministry{Министерство науки и высшего образования}
\country{Российской Федерации}
\fulluniversityname{ФГБОУ ВО Алтайский государственный университет}
\institute{Институт цифровых технологий, электроники и физики}
\department{Кафедра вычислительной техники и электроники}
\departmentchief{В.\,В.~Пашнев}
\departmentchiefdegree{к.ф.-м.н., доцент}
\shortdepartment{ВТиЭ}
\abstractRU{Пока счётчик работает не правильно! Поправьте количество рисунков и таблиц в cls-файле.}
\keysRU{Курсовая работа, рост дендрита, объектно-ориентированное программирование}
\keysEN{computer simulation, distributed version control}

\date{\the\year}

% Подключение файлов с библиотекой.
\addbibresource{graduate-students.bib}

% Пакет для отладки отступов.
%\usepackage{showframe}

\begin{document}
\maketitle

\setcounter{page}{2}
\makeabstract
\tableofcontents

\chapter*{Введение}
\phantomsection\addcontentsline{toc}{chapter}{ВВЕДЕНИЕ}

\textbf{Актуальность}
\newline
Актуальность данной работы заключается в изучении парадигмы объектно-ориентированного программирования (ООП) для разработки программы, симулирующей рост дендрита. Так как парадигма объекто-ориентированного программирования сейчас используется повсеместно, то её изучение и освоение в будущем поможет в решении различных задач и разработке проектов, ориентированных на ООП.
\newline
\textbf{Цель}
\newline
Разработать программу с графическим интерфейсом, которая симулирует процесс роста дендрита, используя язык программирования Python и его графическую библиотеку TKinter.
\newline
\textbf{Задачи:}
\begin{enumerate}
\item Изучить понятие "Рост дендрита"
\item Изучить и освоить парадигму объекто-ориентированного прогрммирования
\item Изучить и освоить графическую библиотеку TKinter 
\item Разработать программу, симулирующую рост дендрита
\end{enumerate}

% Подключение первой главы (теория):
\include{chapter-1-report-csae.tex}
% Подключение второй главы (практическая часть):
\include{chapter-2-report-csae.tex}
% Подключение третий главы (практическая часть с тестированием:
\include{chapter-3-report-csae.tex}

\chapter{Теоретичекская часть}
В этом разделе мы разберём суть работы, условия и задачи. Также изучим необходимые компоненты и теорию, необходимые для выыполнения работы.
\section{Постановка задачи и условий}
\section{Рост дендрита}
\section{Язык программирования Python}
\section{Объектно-ориентированное программирование}

\chapter{Практическая часть}
В этом разделе мы будем применять полученные знания из теории, которую мы изучили раньше. С помощью этих знаний мы разработаем программу "Рост дендрита" на языке программирования Python


\chapter*{Заключение}
\phantomsection\addcontentsline{toc}{chapter}{ЗАКЛЮЧЕНИЕ}

\begin{enumerate}
\item Пример ссылки на электронный источник~\cite{wikiRUPython;
wikiRUDendrit; wikiRUООП; wikiRUTKinter; wikiRUPyCharm}.
\item Пример ссылки на книгу одного автора~\cite{book1author}.
\item Пример ссылки на книгу 5-ти и более авторов~\cite{book5author}.
\end{enumerate}

\newpage
\phantomsection\addcontentsline{toc}{chapter}{СПИСОК ИСПОЛЬЗОВАННОЙ ЛИТЕРАТУРЫ}
\printbibliography[title={Список использованной литературы}]

\appendix
\newpage
\chapter*{\raggedleft\label{appendix1}Приложение}
\phantomsection\addcontentsline{toc}{chapter}{ПРИЛОЖЕНИЕ}
%\section*{\centering\label{code:appendix}Текст программы}

\begin{center}
\label{code:appendix}Код программы "Рост дендрита"
\end{center}

\begin{minted}[linenos, fontsize=\footnotesize,numbersep=5pt, frame=lines, framesep=2mm]{python}
import tkinter as tk
import random

# Constants
WIDTH, HEIGHT = 1000, 1000  # Size of the glass in cells
PARTICLE_COLOR = 'blue'
BACKGROUND_COLOR = 'white'
NUM_PARTICLES = 1000
PARTICLE_SIZE = 5

class Particle:
    def __init__(self, x, y):
        self.x = x
        self.y = y
        self.docked = False

    def move(self):
        if not self.docked:
            direction = random.choice(['up', 'down', 'left', 'right'])
            if direction == 'up' and self.y > 0:
                self.y -= 1
            elif direction == 'down' and self.y < HEIGHT - 1:
                self.y += 1
            elif direction == 'left' and self.x > 0:
                self.x -= 1
            elif direction == 'right' and self.x < WIDTH - 1:
                self.x += 1

            # Check if the particle has reached the bottom
            if self.y == HEIGHT - 1:
                self.docked = True

class DendriteGrowthApp:
    def __init__(self, root):
        self.root = root
        self.canvas = tk.Canvas(root, width=WIDTH*PARTICLE_SIZE, height=HEIGHT*PARTICLE_SIZE, bg=BACKGROUND_COLOR)
        self.canvas.pack()
        self.particles = [Particle(WIDTH // 2, 0) for _ in range(NUM_PARTICLES)]
        self.docked_particles = set()

    def update(self):
        self.canvas.delete('all')
        for particle in self.particles:
            if particle.docked:
                self.docked_particles.add(particle)
            else:
                particle.move()
                self.canvas.create_rectangle(
                    particle.x * PARTICLE_SIZE, particle.y * PARTICLE_SIZE,
                    (particle.x + 1) * PARTICLE_SIZE, (particle.y + 1) * PARTICLE_SIZE,
                    fill=PARTICLE_COLOR
                )
        self.root.after(100, self.update)  # Schedule the next update

def main():
    root = tk.Tk()
    root.title("Dendrite Growth Simulation")
    app = DendriteGrowthApp(root)
    app.update()
    root.mainloop()

if __name__ == "__main__":
    main()
\end{minted}
\end{document}

